\documentclass[textbook.tex]{subfiles}
\begin{document}

\chapter{Preface}

It becomes quite apparant - after years of studying, browsing various forums, and talking to peers - that the cost of academic books and lecture notes (etc.) is not only too high, but on the rise. In most countries the education is already so expensive, that buying a book just becomes another stress factor as students have to balance a job (often more than one), studying, and are already in a lot of debt.

I am so lucky that i was born in Denmark. University is free, and though the books aren't free, we are literally getting payed to study, so buying books isn't a problem.

As i see students - particularly the knowledge they acquire - as one of the most valuable resources in society, i thought that it was perhaps the stupidest idea in the world to start their working life off with crippling debt.

It is true, academic books are often really expensive, and part of that reason is, that professors - in some countries - are paid by the page count. This results in very long, and quite often, very bad books. Often better alternatives are out there, but since its the professors that gets to choose the corriculum, guess whose book is being used.

So the problem we have to solve, is making an academic book, that is cheap (preferably free), well written, has enough to theory to use as the text for the course, and perhaps most importantly, is well known as an alternative to the expensive text written by the professor.

Perhaps only a few, on even no professors at all will choose this textbook as the textbook for their course, but the students should know that this is an alternative which is free, very good, and which covers the same things as the textbook that the professor wrote.

This seems like a daunting task, perhaps even impossible, and that is why its not done. But hopefully in time, when more people are working on it, it will become a much better textbook.




\end{document}